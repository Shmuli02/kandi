\begin{abstract}

Tutkielman kohteena on kirjautumismenetelmät. Tutkielmassa tarkastellaan minkälaisia yksi- kaksi- ja monivaiheinen kirjautuminen ovat. Tutkielma sisältää käytännön esimerkkejä kirjautumismenetelmien hyödyntämisestä. 

Tutkielmassa todetaan, että yksivaiheiseen kirjautumiseen liittyy turvallisuusongelmia, joita on vaikea paikata tietoturvallisesti. Lisäksi tutkielman kohteena on kaksi- ja monivaiheiseen kirjautumiseen liittyviä turvallisuusongelmia. Tutkielmassa syvennyttiin SMS-varmenteeseen sekä TOTP:seen liittyviin turvallisuusongelmiin, joita yksityiskohtaisesti.

Kirjautumismenetelmien soveltuvuutta vertaillaan erilaisissa digitaalisissa sekä fyysisissä käyttötarkoituksissa. Vertailusta saatujen tulosten perusteella kirjautumismenetelmissä sekä niiden soveltuvuudessa käyttötarkoituksiin on eroavaisuuksia. Verkkopankkitunnukset soveltuvat käyttötarkoituksiin, joissa vaaditaan vahvaa tunnistautumista. Muissa käyttötarkoituksissa vertailun muut kirjautumismenetelmät soveltuvat parhaiten. 
Tutkielman lopussa pohditaan tulevaisuuden kirjautumismenetelmiä. 

Tutkielman perusteella voidaan todeta, että kaksi- ja monivaiheinen kirjautuminen tuo paremman suojan kuin yksivaiheinen kirjautuminen. Kaksi- ja monivaiheista kirjautumista olisi suositeltava käyttää. Vaikka kaksi- ja monivaiheinen kirjautuminen parantavat suojausta liittyy näihinkin kirjautumismenetelmiin turvallisuusongelmia. Kirjautumismenetelmiin liittyy teknisiä haavoittuvuuksia sekä käyttäjän manipulointia kalasteluhyökkäyksissä. 


\end{abstract}
