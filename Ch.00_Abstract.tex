\begin{abstract}

Tutkimuksen kohteena oli kirjautumismenetelmät. Tutkielman aluksi selvitettiin mitä yksi- kaksi- ja monivaiheinenkirjautuminen ovat. Millaisista todentamistavoista nämä kirjautumismenetelmät koostuvat sekä millaisia käytännön kirjautumismenetelmiä on olemassa.

Tutkimuksen todettiin, että yksivaiheiseenkirjautumiseen liittyy monia turvallisuusongelmia. Tutkimuksessa myös tutkittiin kaksi- ja monivaiheiseenkirjautumiseen liittyviä turvallisuus ongelmia. Tutkimuksessa selvisi, että kaksi- ja monivaiheiseenkirjautumiseen liittyy myös monia turvallisuus ongelmia sekä riskejä. Tutkimukseen valittiin muuttama turvallisuusongelma, jotka käytiin tarkemmin läpi.

Turvallisuusongelmien jälkeen tutkimuksessa laadittiin vertailu kirjautumismenetelmien soveltuvuudesta eri käyttötarkoituksissa. Vertailussa vertailtiin kirjautumismenetelmien soveltuvuutta digitaalisissa sekä fyysisissä käyttötarkoituksissa. Vertailusta saatujen tulosten perusteella kirjautumismenetelmissä sekä niiden soveltuvuudessa käyttötarkoituksiin on eroavaisuuksia. Kirjautumismenetelmien vertailun jälkeen pohdittiin mahdollisia tulevaisuuden kirjautumismenetelmiä. Pohdinnassa ilmeni monia mahdollisia tulevaisuuden kirjautumismenetelmiä.

Tutkimuksen perusteella voidaan todeta, että kaksi- ja monivaiheinenkirjautuminen tuo paremman suojan kuin yksivaiheinenkirjautuminen. Kaksi- ja monivaiheistakirjautumista olisi suositeltava käyttää. Vaikka kaksi- ja monivaiheinenkirjautuminen parantavat suojausta liittyy näihinkin kirjautumismenetelmiin turvallisuusongelmia. 

\end{abstract}
