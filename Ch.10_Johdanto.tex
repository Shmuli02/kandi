\chapter{Johdanto\label{johdanto}}

Digitaalisten palveluiden käyttö on lisääntynyt viimeisten vuosien aikana. Palveluiden siirtyessä enemmissä määrin Internetiin kasvaa turvallisuuden ja todentamisen merkitys. Pelkästään oikean henkilön tulisi päästä omiin tietoihinsa ja muilta pääsy tulisi olla estettävissä. Oikeutetun henkilön pääsyn varmistamiseksi käytetään kirjautumista. Kirjautumismenetelmiä ja toteutuksia on olemassa monia erilaisia ja ne kehittyvät jatkuvasti. Aikaisemmin kirjautuminen toteutettiin pelkästään yksivaiheiseen todentamiseen perustuen. Tästä on siirrytty käyttämään kaksi- ja monivaiheista kirjautumista, jotka parantavat turvallisuutta \citep{cryptography2010001}.

LastPass verkkosivulla julkaistun The Psychology of Passwords tutkimuksen mukaan 55 \% ihmisistä tietää, että turvallisen salasanan käyttäminen on tärkeää. 91 \% ymmärtää riskit liittyen saman salasanan käyttämiseen useissa palveluissa. Silti 61 \% ihmisistä käyttää samaa tai samalaista salasanaa useassa palvelussa, koska erilaisten salasanojen muistaminen on työlästä \cite{lastpass}. Kaksi- ja monivaiheisen kirjautumisen käyttäminen auttaa käyttäjätunnusten suojaamista. Kaksi- ja monivaiheisen kirjautumisen käyttäminen on yksi parhaimmista menetelmistä turvallisuuden parantamiseksi \citep{top_security_practices}. Kaksi- ja monivaiheisen kirjautumisen käyttämisellä on vaikutusta ihmisten turvallisuuden tunteeseen. Kyselyn mukaan 86 \% kokee, että kaksi- ja monivaiheisen kirjautumisen käyttäminen saa tuntemaan, että omat tiedot ovat paremmin suojattu ja turvassa \citep{nist_2fa}. Kaksi- sekä monivaiheisella kirjautumisella on teknisiä turvallisuutta parantavia sekä turvallisuuden tunteen parantavia vaikutuksia.

Tutkielmassa arvioidaan kirjautumismenetelmiä. Tavoitteena on selvittää, millaisia todentamismenetelmiä kirjautumismenetelmissä käytetään. Tavoite on selvittää, mitä yksi-, kaksi- ja monivaiheinen kirjautumiset tarkoittavat ja millaisia toteutustapoja näistä on olemassa. Tavoitteena on selvittää millaisia turvallisuusongelmia ja riskejä kirjautumismenetelmiin liittyy. Tutkielmassa tehdään kirjautumismenetelmien vertailu, joka selvittää kirjautumismenetelmien soveltuvuutta erilaisissa käyttötarkoituksissa. Lisäksi pohditaan tulevaisuuden kirjautumisvaihtoehtoja.

Tutkielman luvussa 2 esitellään erilaiset todentamistavat sekä erivaiheiset kirjautumismenetelmät. Luvussa 3 perehdytään kirjautumismenetelmien turvallisuusongelmiin. Luvussa 4 tutkitaan kirjautumismenetelmien soveltuvuutta erilaisissa käyttötarkoituksissa. Luvussa 5 pohditaan mahdollisia tulevaisuuden kirjautumismenetelmiä.



