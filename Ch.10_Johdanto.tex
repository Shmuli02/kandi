\chapter{Johdanto\label{johdanto}}

Digitaalisten palveluiden käyttö on lisääntynyt viimeisten vuosien aikana. Palveluiden siirtyessä enemmissä määrin Internetiin kasvaa turvallisuuden ja todentamisen merkitys. Pelkästään oikean henkilön tulisi päästä omiin tietoihinsa ja muilta pääsy tulisi olla estettävissä. Oikeutetun henkilön pääsyn varmistamiseksi käytetään kirjautumista. Kirjautumismenetelmiä ja toteutuksia on olemassa monia erilaisia ja ne kehittyvät jatkuvasti. Aikaisemmin kirjautuminen toteutettiin pelkästään yksivaiheiseen todentamiseen perustuen. Tästä on siirrytty käyttämään kaksi- ja monivaiheistakirjautumista, jotka parantavat turvallisuutta \citep{cryptography2010001}.

LastPass verkkosivulla julkaistun The Psychology of Passwords tutkimuksen mukaan 55 \% ihmisistä tietää, että turvallisen salasanan käyttäminen on tärkeää. 91 \% ymmärtää riskit liittyen saman salasanan käyttämiseen useissa palveluissa. Silti 61 \% ihmisistä käyttää samaa tai samalaista salasanaa useassa palvelussa, koska erilaisten salasanojen muistaminen on työlästä \cite{lastpass}. Kaksi- ja monivaiheisenkirjautumisen käyttäminen auttaa käyttäjätunnusten suojaamista. Kaksi- ja monivaiheisenkirjautumisen käyttäminen on yksi parhaimmista menetelmistä turvallisuuden parantamiseksi \citep{top_security_practices}. Kaksi- ja monivaiheisenkirjautumisen käyttämisellä on vaikutusta ihmisten turvallisuuden tunteeseen. Kyselyn mukaan 86 \% kokee, että kaksi- ja monivaiheisenkirjautumisen käyttäminen saa tuntemaan, että omat tiedot ovat paremmin suojattu ja turvassa \citep{nist_2fa}. Kaksi- sekä monivaiheisellakirjautumisella on teknisiä turvallisuutta parantavia sekä turvallisuuden tunteen parantavia vaikutuksia.

Tutkielmassa arvioidaan kirjautumismenetelmiä. Tavoitteena on selvittää, millaisia todentamismenetelmiä kirjautumismenetelmissä käytetään. Tavoite on selvittää, mitä yksi-, kaksi- ja monivaiheinenkirjautumiset tarkoittavat ja millaisia toteutustapoja näistä on olemassa. Tavoitteena on selvittää millaisia turvallisuusongelmia ja riskejä kirjautumismenetelmiin liittyy. Tutkielmassa tehdään kirjautumismenetelmien vertailu, joka selvittää kirjautumismenetelmien soveltuvuutta erilaisissa käyttötarkoituksissa. Tutkielman tavoitteena on pohtia tulevaisuuden kirjautumisvaihtoehtoja.

Tutkielman kirjautumismenetelmät luvussa esitellään erilaiset todentamistavat sekä erivaiheiset kirjautumismenetelmät. Kirjautumismenetelmien turvallisuus kappaleessa perehdytään kirjautumismenetelmien turvallisuusongelmiin. Vertailu luvussa tutkitaan kirjautumismenetelmien soveltuvuutta erilaisissa käyttötarkoituksissa. Tulevaisuus luvussa pohditaan mahdollisia tulevaisuuden kirjautumismenetelmiä.

