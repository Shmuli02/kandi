\chapter{Johdanto\label{johdanto}}

Kaksivaiheinen kirjautuminen on yleistynyt käyttäjätunnusten suojaamiseksi. Sen avulla pelkkä salasanan tietäminen ei riitä kirjautumiseen vaan tarvitaan myös muita tapoja kirjautumisen varmistamiseksi. Kaksivaiheisia tunnistautumistoteutuksia on olemassa useita erilaisia. Eri toteutuksilla on hyviä ja huonoja puolia käytettävyyteen ja turvallisuuteen liittyen.

Tutkimuksen mukaan 55 \% ihmisistä tietää, että turvallisen salasanan käyttäminen on tärkeää. 91 \% ymmärtää riskit liittyen saman salasanan käyttämiseen useissa palveluissa. Mutta silti 61 \% ihmisistä käyttää samaa tai samalaista salasanaa useassa palvelussa. \cite{lastpass}

Tutkielmassa käsitellään erilaiset kirjautumismenetelmät siihen liittyvät turvallisuus ongelmat. Näiden pohjalta vertailu kappaleessa käydään läpi kirjautumismenetelmä eri tilanteissa. Lopuksi pohditaan kirjautumismenetelmien tulevaisuutta. 
