\chapter{Johdanto\label{johdanto}}

Digitalisaatio ja digitaalistenpalvelut ovat kasvaneet viimeisten vuosien aikana. Palveluiden siirtyessä enemmissä määrin internettiin kasvaa turvallisuuden ja varmentamisen merkitys. Pelkästään oikean henkilön tulisi päästä omiin tietoihin ja muilta pääsy tulisi olla estetty. Pääsyn varmistamiseksi käytetään kirjautumista. Kirjautumismenetelmiä ja toteutuksia on olemassa monia erilaisia. Kirjautumismenetelmät ovat kehittyneet sekä kehittyvät jatkuvasti. Aikaisemmin kirjautuminen toteutettiin pelkästään yksivaiheiseen todentamiseen perustuen. Tästä on siirrytty käyttämään kaksi- ja monivaiheistakirjautumista, jotka parantavat turvallisuutta. \citep{cryptography2010001}

Tutkimuksen mukaan 55 \% ihmisistä tietää, että turvallisen salasanan käyttäminen on tärkeää. 91 \% ymmärtää riskit liittyen saman salasanan käyttämiseen useissa palveluissa. Mutta silti 61 \% ihmisistä käyttää samaa tai samalaista salasanaa useassa palvelussa. \cite{lastpass} Kaksi- ja monivaiheisenkirjautumisen käyttäminen auttaa käyttäjätunnusten suojaamista. Kaksi- ja monivaiheisenkirjautumisen käyttäminen on yksi parhaimmista menetelmistä turvallisuuden parantamiseksi.  \citep{top_security_practices} Kaksi- ja monivaiheisenkirjautumisen käyttämisellä on myös ---- vaikutusta. Kyselyn mukaan 86 \% kokee, että kaksi- ja monivaiheisenkirjautumisen käyttäminen saa tuntemaan, että omat tiedot ovat paremmin suojattu ja turvassa. \citep{nist_2fa} Kaksi- sekä monivaiheisellakirjautumisella on teknisiä turvallisuutta parantavia sekä myös turvallisuuden tunteen parantavia vaikutuksia.

Tutkimuksen tavoitteena on tutkia kirjautumismenetelmiä. Tavoitteena on selvittää, millaisia todentamismenetelmiä kirjautumismenetelmissä käytetään. Tavoite on selvittää mitä yksi- kaksi ja monivaiheinenkirjautuminen tarkoittaa ja millaisia toteutustapoja näistä on olemassa. Tutkimuksen tavoita on myös selvittää ja tutkia millaisia turvallisuusongelmia ja riskejä kirjautumismenetelmiin liittyy. Tutkimuksessa tehdään kirjautumismenetelmien vertailu. Tavoitteena on selvittää kirjautumismenetelmien soveltuvuutta erilaisissa käyttötarkoituksissa. Tutkimuksen lopussa on tavoite myös pohtia tulevaisuuden kirjautumisvaihtoehtoja. 

