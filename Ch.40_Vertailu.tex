\chapter{Vertailu\label{vertailu}}

Tässä kappaleessa vertaillaan kirjautumismenetelmien soveltuvuutta erilaisissa käyttöskenaarioissa.Vertailuun on valittu seitsemän erilaista käyttötarkoitusta, joissa kaksivaiheistakirjautumista voitaisiin käyttää.
\begin{itemize}
    \item Verkkosivut
    
    Tällä tarkoitetaan yleisesti erilaisia verkossa olevia sivuja ja palveluita.
    \item Pankki palvelut
    
    Pankki palvelut ovat yksi kriittisimmistä ja tärkeimmistä palveluista. Tämän takia pankki palveluiden suojaus täytyy myös olla korkealla tasolla
    \item Sähköposti
    \item Julkiset palvelut
    
    Tällä tarkoitetaan julkisia palveluja netissä kuten Suomi.fi, Omakanta, Omavero
    \item Tietokone
    \item Yle areena
    \item IoT laitteet
\end{itemize}

Vertailuun on valittu kuusi erilaista varmentamismenetelmää. Valitut varmentamismenetelmät ovat SMS varmenne, Kertakäyttöiset koodit, Fyysinen avain, Mobiilisovellus, Sähköposti ja Verkkopankkitunnukset. 


Varmennusmenetelmien soveltuvuutta arvioidaan viisi portaisella asteikolla nollasta neljään. Nolla tarkoittaa, että kyseinen varmentamismenetelmä ei sovellu lainkaan käyttötarkoitukseen. Ykkönen tarkoittaa, että menetelmä sopii mutta ei ole suositeltava. Kaksi tarkoitta, että mahdollinen käytettävä varmentamismenetelmä mutta siihen saattaa liittyä joitakin ongelmia. Arvosana kolme tarkoittaa, että suositeltava. Kyseistä varmentamismenetelmää suositellaan käytettäväksi. Ja korkein arvosana neljä tarkoittaa, että kyseinen varmentamismenetelmä on vahvasti suositeltava ja sitä tulisi käyttää.


\begin{table}[ht]
\begin{tabular}{ |p{3cm}|p{1cm}|p{1,5cm}|p{1,5cm}|p{1,5cm}|p{2cm}|p{2,5cm}|  }
 \hline
 \multicolumn{7}{|c|}{ Kirjautumismenetelmät} \\
 \hline
 & SMS & TOTP &Fyysinen avain & Mobiili-sovellus & sähköposti & Verkkopankki-tunnukset\\
 \hline
 Verkkosivut& 2 & 2 & 3 & 3 & 2 &  1\\
 Pankki palvelut& 1 & 3 & 3 & 3 & 1 &  4\\
 Sähköposti& 2 & 2 & 2 & 2 & 2 & 1 \\
 Julkiset palvelut& 2 & 2 & 3 & 2 & 2 &  3\\
 Tietokone& 0 & 0 & 2 & 0 & 0 &  1\\
 Yle areena& 2 & 2 & 2 & 2 & 2 &  1\\
 IoT laitteet& 2 & 2 & 2 & 2 & 2 &  0\\
 \hline
\end{tabular}
\caption{\label{tab:vertailu} Kirjautumismenetelmien vertailu}
\end{table}

0: ei sovellu
1: ei suositeltava
2: mahdollinen
3: suositeltava
4: vahva suositus
