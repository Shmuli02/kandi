\chapter{Tulevaisuus\label{tulevaisuus}}

Tässä kappaleessa pohditaan kirjautumismenetelmien tulevaisuutta. Mikä on nykyisten kirjautumismenetelmien tulevaisuus. Mitä mahdollisia uusia kirjautumismenetelmiä tulevaisuudessa saatetaan kehittää ja käyttää.


Biometristen varmentamismenetelmissä saatetaan nähdä kehitystä ja uusia menetelmiä tulevaisuudessa
EKG signaalin käyttäminen biometrisessä varmentamisessa on yksi mahdollinen tulevaisuuden vaihtoehto. Varmentaminen perustuu EKG signaaliin, joka on yksilöllinen. EKG signaalin käyttämisestä tekee mielenkiintoisen se, että sen kopiointi tai manipulointi on lähes mahdotonta. EKG:tä voisi olla mahdollinen biometrinen varmentamismenetelmä. \citep{shdefat2018utilizing}
Monivaiheinen biometrinen varmentaminen on mahdollinen tulevaisuuden varmentamismenetelmä. Menetelmä perustuisi kahteen tai useampaan biometriseen tunnistuksen. Tämän avulla pystyttäisiin tunnistamaan ja varmentamaan henkilöllisyys paremmin. \citep{biometric_authentication_systems}

Passiivista biometrista dataa käyttäminen varmentamisessa olisi myös mahdollista. Passiivinen biometrinen todentaminen on aktiivisen biometrisen todentamisen vastakohta. Se ei vaadi käyttäjän panosta esimerkiksi sormen laittamista laitteeseen, joka lukee sormenjäljen. Passiivista biometrista dataa voi olla esimerkiksi henkilön tapa kävellä, millä tavalla pitää ja käyttää puhelinta, millä tavalla puhuu, kehon lämpötila, kehon ääriviivat. Nykyään käytössä oleva kasvojen tunnistaminen voidaan laskea passiiviseksi tunnistamiseksi. Se ei vaadi käyttäjältä panosta varmentamiseen. \citep{biometric_authentication_systems}\citep{passive_biometrics}

Vaikka vahva tunnistautuminen on tärkeää, tulisi pitää jokin järkevä raja kuinka monimutkainen varmentamisprosessi on. Esimerkiksi Korean pankilla on käytössä monimutkainen ja monivaiheinen kirjautuminen. Käyttäjän täytyy asentaa tietokoneelleen ohjelman, joka sisältää sertifikaatin kirjautumista varteen. Käyttäjä tarvitsee myös fyysisen numerokortin, jossa on kertakäyttöisiä koodeja. Näiden lisäksi vielä PIN koodi, salasana ja puhelin tekstiviestillä lähetettävää kertakäyttöistä koodia varten.
Kirjautuminen koostuu siis kahdesta asiasta, jotka tiedä (PIN ja salasana). Ja tämän lisäksi kolmesta asiasta, jotka omistat (sertifikaatti, koodi kortti ja SMS). \citep{rittenhouse2015survey}
