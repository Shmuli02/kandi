\chapter{Tulevaisuus\label{tulevaisuus}}

Tässä kappaleessa perehdytään kirjautumismenetelmien tulevaisuuteen. Kappaleessa pohditaan uusia kirjautumiseen käytettäviä menetelmiä tulevaisuudessa olisi mahdollista käyttää kirjautumisessa. Kappaleessa tutustutaan salasanattomaan kirjautumiseen.

Biometristen varmentamismenetelmissä saatetaan nähdä kehitystä ja uusia menetelmiä tulevaisuudessa EKG-signaalin käyttäminen biometrisessä varmentamisessa on yksi mahdollinen tulevaisuuden vaihtoehto. Varmentaminen perustuu EKG-signaaliin, joka on yksilöllinen. EKG-signaalin käyttämisestä tekee mielenkiintoisen se, että sen kopiointi tai manipulointi on lähes mahdotonta. EKG:tä voisi olla mahdollinen biometrinen varmentamismenetelmä \citep{shdefat2018utilizing}.

Monivaiheinen biometrinen varmentaminen on mahdollinen tulevaisuuden varmentamismenetelmä. Menetelmä perustuisi kahteen tai useampaan biometriseen tunnistuksen. Tämän avulla pystyttäisiin tunnistamaan ja varmentamaan henkilöllisyys paremmin \citep{biometric_authentication_systems}.

Passiivista biometrista dataa käyttäminen varmentamisessa olisi myös mahdollista. Passiivinen biometrinen todentaminen on aktiivisen biometrisen todentamisen vastakohta. Se ei vaadi käyttäjän panosta esimerkiksi sormen laittamista laitteeseen, joka lukee sormenjäljen. Passiivista biometrista dataa voi olla esimerkiksi henkilön tapa kävellä, millä tavalla pitää ja käyttää puhelinta, millä tavalla puhuu, kehon lämpötila, kehon ääriviivat. Nykyään käytössä oleva kasvojen tunnistaminen voidaan laskea passiiviseksi tunnistamiseksi. Se ei vaadi käyttäjältä panosta varmentamiseen \citep{biometric_authentication_systems} \citep{passive_biometrics}.

