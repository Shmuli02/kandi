\chapter{Yhteenveto\label{yhteenveto}}

Tutkielmassa arvioitiin erilaisia kirjautumismenetelmiä. Tavoitteena oli selvittää kirjautumismenetelmien turvallisuutta erilaisissa käyttötarkoituksissa.

Tutkielmassa laadittiin kirjautumismenetelmien analyysi. Analyysin perusteella voidaan todeta, että verkkopankkitunnukset soveltuvat parhainten verkkopankkeihin sekä julkisien palveluiden kirjautumiseen, joissa vaaditaan vahvaa tunnistautumista. Muihin käyttötarkoituksiin verkkopakkitunnuksia ei ole suositeltavaa käyttää, koska tunnuksia ei tule altistaa muhin käyttötarkoituksiin riskien välttämiseksi.

Teknologian kehittyessä myös kirjautumistavat kehittyvät. Teknologian kehittyessä tulevaisuudessa saatetaan nähdä uusia kehitysaskelia. Tulevaisuudessa saatetaan nähdä uudenlaisia teknologioita ja mahdollisesti parempia kirjautumismenetelmiä. Tutkielmassa todettiin, että passiivista biometristä dataa sekä muita uusia biometrisiä todentamistapoja olisi mahdollista käyttää tulevaisuudessa kirjautumismenetelmänä. Yhtenäisten kirjautumisjärjestelmien mahdollisuudesta ja sen hyödyntämistä tulevaisuudessa esiteltiin tutkielmassa pääpiirteittäin. 

Yhteenvetona voidaan todeta, että yksivaiheinen kirjautuminen ei ole turvallinen vaihtoehto, koska siihen liittyy monia turvallisuusongelmia. Kaksi- ja monivaiheisen kirjautumisen käyttäminen on suositeltavaa sillä se parantaa käyttäjätilien turvallisuutta. Käyttäjien tulee muistaa, että kaikkiin kirjautumismenetelmiin liittyy turvallisuusongelmia. 
