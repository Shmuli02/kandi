\chapter{Yhteenveto\label{yhteenveto}}

Tutkimuksen kohteena oli kirjautumismenetelmät. Tutkimuksen tavoitteena oli selvittää, millaisia kirjautumismenetelmiä on olemassa sekä kirjautumismenetelmien soveltuvuutta erilaisissa käyttötarkoituksissa. Tutkimuksessa tutkittiin kirjautumistapoja, niiden eroavaisuuksia, turvallisuusongelmia sekä soveltuvuutta. Tutkimuksen tavoite oli ylisellä tasolla ymmärtää kirjautumismenetelmien eroavaisuuksia, hyviä puolia sekä heikkouksia.

Tutkimuksessa laadittiin kaksi- ja monivaiheistenkirjautumistapojen vertailu. Vertailussa tutkittiin kirjautumistapojen soveltuvuutta erilaisissa käyttötarkoituksissa. Vertailussa havaittiin, että kirjautumistapojen soveltuvuudessa on eroavaisuuksia. Vertailusta saatujen tulosten perusteella huomattiin, että verkkopankkitunnukset soveltuvat parhainten verkkopankkeihin sekä julkisien palveluiden kirjautumiseen, joissa vaaditaan vahvaa tunnistautumista. Muihin käyttötarkoituksiin verkkopakkitunnukset voisivat soveltua, mutta ei ole suositeltavaa liittyen turvallisuuteen sekä tunnusten tärkeyteen. Tutkimuksen vertailussa muut vertailussa olleet kirjautumistavat soveltuivat tasaisesti vertailun muihin käyttötarkoituksiin.

Teknologian kehittyessä myös kirjautumistavat kehittyvät. Yksivaiheisesta kirjautumistavasta ensimmäinen askel oli kaksivaiheinenkirjautuminen. Kaksivaiheisestakirjautumisesta seuraava askel oli monivaiheinenkirjautuminen. Teknologian kehittyessä tulevaisuudessa saatetaan nähdä uusia kehitysaskelia. Tulevaisuudessa saatetaan nähdä uudenlaisia sekä mahdollisesti parempia kirjautumismenetelmiä. Tutkielmassa pohdittiin, että passiivista biometristä dataa sekä muita uusia biometrisiä todentamistapoja olisi mahdollista käyttää tulevaisuudessa kirjautumismenetelmänä. Myös yhtenäisten kirjautumisjärjestelmien mahdollisuutta pohdittiin tutkielmassa.

Tutkimuksen perusteella voidaan todeta, että yksivaiheinenkirjautuminen ei ole turvallinen vaihtoehto. Siihen liittyy monia turvallisuusongelmia. Kaksi- ja monivaiheisen kirjautumisen käyttäminen on suositeltavaa. Ne parantavat käyttäjätilien turvallisuutta, mutta myös näihinkin menetelmiin liittyy turvallisuusongelmia. 
